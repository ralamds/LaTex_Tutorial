\documentclass[12pt]{book}
%%%%%%%%%%%%%%%%%%%%%%%%%%%%%%
% Preambulo
%%
\usepackage[T1]{fontenc}
\usepackage[utf8]{inputenc}
\usepackage[spanish]{babel}
%%
%\parindent = 0cm
\begin{document}
\title{Practica 2.\\ Texto normal, párrafos y alineación }
\author{Héctor Misael}
\date{}
\maketitle
\tableofcontents
\chapter{Texto}
\section{Tipos y tamaños de fuente}
Iniciamos con la segunda practica de este tutorial, colocando el titulo autor y dia. Esta es la primera sección donde conoceremos algunos tipos de letra, asi como los diferentes tamaños de letra que posemos usar.
\subsection{Tipos de letra}
\begin{itemize}
\item Tipo de letra \textbf{negrita}
\item Tipo de letra \textit{itálica}
\item Tipo de letra \textrm{romana}
\item Tipo de letra \textsf{sans serif}
\item Tipo de letra \texttt{mono espaciada}
\item Tipo de letra \textsl{inclinada}
\item Tipo de letra \textsc{versalitas}
\end{itemize}

\newpage

\subsection{Tamaños de letra}
\begin{itemize}
\item {\tiny Tamaño} de letra
\item {\scriptsize Tamaño} de letra
\item {\footnotesize Tamaño} de letra
\item {\small Tamaño} de letra
\item {\normalsize Tamaño} de letra
\item {\large Tamaño} de letra
\item {\Large Tamaño} de letra
\item {\LARGE Tamaño} de letra
\item {\huge Tamaño} de letra
\item {\Huge Tamaño} de letra
\end{itemize}

\section{Párrafos, sangría y saltos de linea.}
\subsection{Sangría}
\noindent Esta es la primera sección donde conoceremos algunos tipos de letra, así como los diferentes tamaños de letra que posemos usar. Y ahora queremos también decidir y modificar la sangria en el documento, aqui si iniciamos con sangría.

\noindent Pero esta otra linea queremos que no tenga sangria

Pero esta otra linea queremos que si tenga sangria
\subsection{Salto de linea y nueva pagina}
Ahora queremos también decidir y modificar la sangria en el documento, aquí si iniciamos con sangría.\\
Pero esta otra linea queremos que no tenga sangría\par 
Esta linea queremos que si tenga sangría y ademas un espacio mayor.\\[0mm]
Pero esta otra linea queremos que no tenga sangría.
\subsection{Alineación de párrafos}
\begin{flushleft}
Texto a la izquierda
\end{flushleft}

\begin{center}
Texto centrado
\end{center}

\begin{flushright}
Texto a la derecha
\end{flushright}

\subsection{Comillas y puntos suspensivos}
Este es un ejemplo: ``para las comillas'',  ` simples ', y las comillas dobles ``dobles''.\\
Y tambien los puntos suspensivos \dots

\subsection{Espaciado horizontal}
\noindent Inicio \, fin\\[0.2cm]
Inicio \quad fin\\[0.2cm]
Inicio \qquad fin\\[0.2cm]
Inicio \hspace{3cm} fin\\[0.2cm]
Inicio \hspace{5cm} fin\\[0.2cm]
Inicio \hfill fin\\[0.2cm]
\subsection{Lineas de relleno}
\noindent Inicio \hfill fin\\[0.5cm]
Inicio \hrulefill fin\\[0.5cm]
Inicio \dotfill fin

\end{document}


