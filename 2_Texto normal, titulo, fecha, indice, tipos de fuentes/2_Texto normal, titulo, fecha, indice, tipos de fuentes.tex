\documentclass[12pt]{book}
%%
% Preambulo
%%
\usepackage[T1]{fontenc}
\usepackage[utf8]{inputenc}
\usepackage[spanish]{babel}
%%
%\parindent = 1cm
\begin{document}
\title{Practica 2.\\ Texo normal, parrafos y alineacion}
\author{Rakibul Alam}
\date{19-10-2019}
\maketitle

\tableofcontents

\chapter{Texto}

\section{Tipos y tamanos de fuente}
Indicamos con la segunda practica de este tutorial, colocando el titulo autor y dia. Esta es la primera seccion donde conocemos algunos tipos de letra que podemos usar. 

\subsection{Tipos de letra}
\begin{itemize}
\item Tipo de letra: \textbf{Negra}
\item Tipo de letra: \textit{Italia Pasta}
\item Tipo de letra: \textrm{romana}
\item Tipo de letra: \textsf{Sans Serif}
\item Tipo de letra: \texttt{mono espaciada}
\item Tipo de letra: \textsl{inclinada}
\item Tipo de letra: \textsc{versalitas}
\end{itemize}

\subsection{Tamanos de letra}
\begin{itemize}
\item {\tiny tiny Tamano} de letra
\item {\scriptsize scriptsizeTamano} de letra
\item {\footnotesize footnotesize Tamano} de letra
\item {\small small Tamano} de letra
\item {\normalsize normalsize Tamano} de letra
\item {\large large Tamano} de letra
\item {\Large Large Tamano} de letra
\item {\LARGE LARGE Tamano} de letra
\item {\Huge HUGE Tamano} de letra
\end{itemize}

\section{Parrafos, sangra y saltos de linea.}
Aqui vamos a ver como se hace los parrafos, sangria y saltos de linea. 

\subsection{Sangria}
En esta parte aprnderemos la sangria. 
Que es la sangria? pues es un desplazamiento. \\
\noindent si usamos "noindent" pues se elimina. 

\subsection{Salto de linea y nuava pagina}
Ahora a aprnder salto de linea y nueve pagina. \\
con dos slash se salta una linea. 
\par esta linea va a tener sangria y nueva linea.\\[0.5mm]
Hahah aqui es otra linea que no tenga sangria pero 0.5 mas espacio mas. 
 
\subsection{Alineacion de parrafos}
\begin{flushleft}
Texto de izquierda
\end{flushleft}

\begin{center}
Texto al centro
\end{center}

\begin{flushright}
Texto a la derecha
\end{flushright}

\subsection{Comillas y puntos suspensivos. }
Este es un ejemplo: "para las comillas", 'simples' y \\ puntos suspensivos \dots 

\subsection{Espacio horizontal}
Inicio \, fin\\[0.2cm]
Inicio \quad fin\\[0.2cm]
Inicio \qquad fin\\[0.2cm]
Inicio \hspace{2cm} fin\\[0.2cm]
Inicio \hfill fin\\[0.2cm]

\subsection{Lineas de relleno}
Inicio \hfill fin\\[0.2cm]
Inicio \hrulefill fin\\[0.2cm]
Inicio \dotfill fin\\[0.2cm]


\end{document}

