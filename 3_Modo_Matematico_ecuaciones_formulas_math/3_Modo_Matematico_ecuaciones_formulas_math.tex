\documentclass[12pt]{article}

% Preambulo

\usepackage[T1]{fontenc}
\usepackage[utf8]{inputenc}
\usepackage[spanish]{babel}
\parindent = 0cm

\usepackage{amsmath}
\usepackage{amssymb, amsfonts, latexsym}
\usepackage{cancel}


\begin{document}

\title{Practica 3 \\ Modo Matemático, Ecuaciones, Formulas}
\author{Rakibul Alam}
\date{}
\maketitle
\tableofcontents

\section{Texto en modo matemático}
Básicamente hay 2 formas de colocar texto en modo matemáticos               
LaTex, 1. Colocar formulas junto al texto. 2. Colocar de forma independiente. 

\subsection{Formulas junto al texto}
\textbf{Cuadro de binomio.} Sea $ (a+b)^{2} = (a+b) (a+b) $ donde $a$ y $b$ representan números algebraicos cualesquiera, positivos o negativos. Por lo tanto $ (a+b)^{2} = a^{2} + 2ab + b^{2} $.  

\subsection{Formulas independientes}

\textbf{Cuadro de binomio.} Sea 
\[
(a+b)^{2} = (a+b) (a+b)
\]
donde $a$ y $b$ representan números algebraicos cualesquiera, positivos o negativos. Por lo tanto 
\[
(a+b)^{2} = a^{2} + 2ab + b^{2} 
\]

\begin{equation*}
(a+b)^{2} = a^{2} + 2ab + b^{2}
\end{equation*}

\subsection{Numeracion de formulas y referencias}
El comando para numerar una formula es el: 
\begin{equation}\label{binomio uno}
(a+b)^{2} = a^{2} + 2ab + b^{2}
\end{equation}
Haaha here u get some tex. 
\begin{equation}\label{binomio dos}
(x+y)^{2} = x^{2} + 2xy + y^{2}
\end{equation}

Aqui usamos las referencias \ref{binomio uno} y \ref{binomio dos} \\
Aqui usamos las referencias \eqref{binomio uno} y \eqref{binomio dos}

\subsection{Alinear ecuaciones con el comando align}
\begin{align}
x = a+b \\
y = 3c+d \\
z = 4e+5f
\end{align}

\begin{align*}
x &= a+b \\
y &= 2c+d \\
z &= 3e+4f
\end{align*}

\begin{align}
x &= a+b &  x &= a+b \nonumber \\
y &= 2c+d  & y &= 2c+d \\
z &= 3e+4f & z &= 3e+4f
\end{align}

\subsection{Fracciones}
\begin{itemize}
\item Junto a texto $ \frac{x}{y} $
\item Junto a texto $ \dfrac{x}{y} $
\item No junto a texto 
\[
\frac{x}{y}
\]

\end{itemize}

\subsection{Potencias, subindices, superindice}

\begin{itemize}
\item $ x^{6} $
\item $ z_{3}$
\item $ a^{b^{c}}$ 
\end{itemize}

\subsection{Raices}
\begin{itemize}
\item $ \sqrt{a+b}$
\item $ \sqrt[n]{a+c} $
\item \[   \sqrt[n]{c+d}   \]
\end{itemize}

\subsection{Coeficientes binomiales}

\begin{itemize}
\item junto a texto $\binom{n}{k} $
\item \[   \binom{n}{k}  \]

\end{itemize}



\end{document}

