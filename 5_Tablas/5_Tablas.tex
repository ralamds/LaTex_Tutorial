\documentclass[12pt]{article}

\usepackage[T1]{fontenc}
\usepackage[utf8]{inputenc}
\usepackage[spanish,es-tabla]{babel}
\usepackage{amsmath}
\usepackage{amssymb, amsfonts, latexsym, cancel}
\usepackage{graphicx}
\usepackage{epstopdf}
\usepackage{float}
\usepackage{subfigure}

\parindent=0cm

%----------Document--Begin--Here------------%

\begin{document}

\title{Practica 5\\Tablas}
\author{Rakibul Alam}
\date{27-20-2019}
\maketitle
\tableofcontents

%--------End of front page------------------

\section{Incluir tablas básicas}
En \LaTeX \, la forma basica en la que podemos incluir tablas es usando el entorno \textbf{tabular}, su sintaxis es la siguiente:\\[0.3cm]
\noindent \textbf{Formato de las columnas:}
\begin{itemize}
\item l (alineado a la izquierda)
\item c (alineado a la centrada
\item r (alineado a la derecha)
\end{itemize}

\begin{itemize}
\item $\& $ caracter que se utiliza para separar columnas.
\end{itemize}

Nuestra primera tabla. \quad
\begin{tabular}{lcr}
Lunes & Martes & Miercules \\
0 	  &		 0 & 	0\\
1	  &     1  & 1  \\
\end{tabular}

\newpage

\subsection{Entorno table - tablas con lineas}

\begin{table}[!ht]
\centering

\begin{tabular}{ll}
t: & La imagen en la parte superior (top)\\
b: & La imagen en la parte inferior (bottom)\\
h: & La imagen en el sitio que escribimos (here)
\end{tabular}
\caption{Descripcion del entorno table}
\label{Tabla1}
\end{table}

\begin{table}[!ht]
\centering
\begin{tabular}{|l||c|}
Imagen 1 &\includegraphics[scale=.05]{5figuras/wiki.png} \\
Imagen 2 &\includegraphics[scale=.05]{5figuras/wiki.png} \\

\end{tabular}
\end{table}

Aqui hay con linea horizontal y vertical

\begin{table}[!ht]
\begin{tabular}{|l|c|r|}
\hline
Lunes & Martes & Miercules \\
\hline
0 	  &		 0 & 	0\\
1	  &     1  & 1  \\
\hline
\end{tabular}

\end{table}















\end{document}

